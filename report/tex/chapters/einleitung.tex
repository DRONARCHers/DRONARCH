\chapter{Einleitung}
	Ziel der Archäologie ist das Erhalten und Verstehen von materiellen Spuren vergangener Zeit. Dazu gehören Ausgrabungen, die gezielt archäologische Befunde freilegen und diese dabei teilweise zerstören. Die Information, die Befunde liefern können also nicht im Original erhalten werden, sondern müssen laufend dokumentiert werden. Diese Dokumentation bildet die Grundlage für die Interpretation und Auswertung durch heutige und zukünftige Archäologen.
	Das Erfassen dieser Informationen ist dadurch ein ausgesprochen wichtiger Punkt und die kritische Diskussion seiner Methodik verdient einige Aufmerksamkeit.
	
	\section{Klassische Dokumentation}
		Der Archäologische Dienst des Kantons Bern (ADB) schriebt in seinem Handbuch: ''Zeichnungsdokumentation, Beschrieb und Fotodokumentation sind die drei Standbeine der Grabungsdokumentation.'' \cite{adb:handbuch}.
		
		\subsection{Fotodokumentation}
			Fotos bilden visuelle Informationen mit wenig Verfälschung ab. Sie zeigen die Situation auf eine sehr natürliche Weise und sind wenig von Interpretation beeinflusst.
			Beim Fotografieren können die Wahl von Standort, Bildausschnitt, Objektiv und Lichtverhältnis einen gewissen Einfluss auf deren Interpretation haben. Zudem ist die erreichbare Qualität eines Fotos durch die verwendete Ausrüstung beschränkt.
			So können wichtige Details verloren gehen, wenn schlecht Fotografiert werden, oder die Fotos sind schwierig in Kontext zu setzen, wenn es an Übersichtsaufnahmen und Aufnahmedatum fehlt.
			Seit digitale Kameras die analoge Fotografie abgelöst haben, ist die Menge an Foto und deren Verfügbarkeit stark gewachsen (Zahlen zu Flickr: \cite{flickr:number}). Das ist nach E. Gersbach \cite{ausgrabung_heute} im Sinne einer zeitlich und räumlich lückenlosen Dokumentation.
			Doch es führt möglicherweise auch zu einer grossen Menge an wenig informativen Fotos, die insgesamt zwar schneller und einfacher gemacht worden sind, aber letztlich weniger aussagekräftig sind als weniger dafür bessere Fotos.
			Gute Fotos zu machen, daraus eine Auswahl zu treffen und diese mit der restlichen Dokumentation in Kontext zu setzten, kann ein beachtlicher Aufwand sein.
			Automatisierte Unterstützung beim Erfassen und Auswerten könnte diesen Prozess deutlich effizienter machen.

		\subsection{Zeichnungsdokumentation}
			Zu Zeichnungen schreibt der ADB in seinem Handbuch \cite{adb:handbuch}: ''Man kann nur zeichnen was man versteht.''
			Zeichnerische Dokumentation generalisiert, hebt Wichtiges hervor und lässt Unwichtiges weg. Für die Unterscheidung zwischen Wichtig und Unwichtig ist eine Interpretation erforderlich.
			Dies führt dazu, dass das Erstellen der Dokumentation vor Ort mehr Zeit erfordert da der Zeichner zu einer Interpretation gezwungen ist. Somit enthält eine Zeichnung nicht nur die Abbildung eines Befundes, sondern auch eine Interpretation dazu. Dieser reichere Informationsgehalt ist zwar für die spätere Interpretation sehr nützlich, stellt aber auch eine Verfälschung der realen Befundsituation dar.
			
		\subsection{Beschrieb}
			Fotos und Zeichnungen enthalten in erster Linie visuelle Informationen. Weitere wichtige qualitative Merkmale, etwa Materialien und Beschaffenheit, werden in Text festgehalten.
		
			
	\section{Fehlende Dimension}
		Fotos und Skizzen sind zweidimensionale (2D) Beschreibungen einer dreidimensionalen (3D) Realität. Es ist eine Projektion nötig, die um eine Dimension reduziert. Fotografie verwendet naturgemäss eine perspektivische Projektion (Zentralprojektion), für Skizzen hingegen kann auch eine orthographische Projektion (Parallelprojektion) verwendet werden.
		In jedem Fall geht eine ganze Dimension an Informationen verloren und muss mittels Abbildungen verschiedener Perspektiven vom Betrachter rekonstruiert werden.
		Dies erschwert die Interpretation dieser Form von Dokumentation und bedingt eine gute Wahl der Perspektive beim Erstellen der Abbildungen.
		Soweit wird die 3D Realität mittels verschiedener Projektionen und Techniken auf zwei Dimensionen abgebildet und im Kopf des Archäologen wieder in 3D rekonstruiert.
		Deutlich einfacher und intuitiver für den Betrachter ist es die 3D Struktur direkt abzubilden, zu speichern und zu betrachten.