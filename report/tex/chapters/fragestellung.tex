\chapter{Motivation}
	Die Idee 3D Strukturen in drei Dimensionen zu Erfassen und Darzustellen ist naheliegend, bedingt aber Aufnahmegeräte und eine Datenrepräsentation, die 3D Inhalte unterstützen. Dazu kommen neben aufwändigen Miniaturen nur computergestützte Verfahren in Frage.
	
	\section{Ziel dieser Arbeit} \label{frag:ziel}
		In dieser Arbeit werden Verfahren zur 3D Dokumentation präsentiert und deren technischen \emph{Möglichkeiten} und \emph{Nützlichkeit} für die Archäologie diskutiert.
		Insbesondere wird \emph{Structure from Motion} (SfM, siehe \autoref{sfm}) als einfaches und günstiges Verfahren zum Erstellen von 3D Modellen betrachtet und mit anderen Methoden verglichen.
		Anhand einiger Fallstudien soll die Möglichkeit der Integration in bestehende Grabungs- und Dokumentationspraxis und deren Vor- und Nachteile überprüft werden.
		Die Hauptpunkte sind dabei:
		\begin{itemize}
			\item
			\emph{Mehrwert} der 3D Dokumentation gegenüber der 2D Dokumentation
			\item
			\emph{Aufwand} zum Erstellen eines 3D Modells
			\item
			\emph{Qualität} und Fehler eines gewonnen Modells
			\item
			\emph{Integration} anderer Dokumentationen, insbesondere GIS-Daten
		\end{itemize}
		
	\section{DRONARCH}
		Als technische Grundlage für die Fallstudien dient die Software \emph{DRONARCH} \cite{dronarch:github}, die im Rahmen dieser Arbeit geschrieben wurde und das SfM Verfahren verwendet.
		Im Gegensatz zu bestehenden Publikationen, die sich mit SfM in der Archäologie beschäftigen \cite{arch:laser_vs_dense_stereo, ARCM:ARCM667, ARP:ARP399, TUW-210216, DeReu20131108, the_cave, altai}, verwendet und publiziert \emph{DRONARCH} nur Open Source Software und will ein Werkzeug bieten mit wenig Geld und Zeitaufwand 3D Modell für den archäologischen Einsatz zu erstellen.

	