\chapter{Motivation}
	Die Idee 3D Strukturen in drei Dimensionen zu Erfassen und Darzustellen ist naheliegend, bedingt aber Aufnahmegeräte und eine Datenrepräsentation, die 3D Inhalte unterstützen. Dazu kommen neben aufwändigen Miniaturen nur computergestützte Verfahren in Frage.
	
	\section{Ziel dieser Arbeit} \label{frag:ziel}
		In dieser Arbeit werden Verfahren zur 3D Dokumentation präsentiert und deren technischen \emph{Möglichkeiten} und \emph{Nützlichkeit} für die Archäologie diskutiert.
		Insbesondere wird \emph{Structure from Motion} (SfM, siehe \autoref{sfm}) als einfaches und günstiges Verfahren zum Erstellen von 3D Modellen betrachtet und mit anderen Methoden verglichen.
		Anhand einiger Fallstudien soll die Möglichkeit der Integration in bestehende Grabungs- und Dokumentationspraxis und deren Vor- und Nachteile überprüft werden.
		Die Hauptpunkte sind dabei:
		\begin{itemize}
			\item
			\emph{Mehrwert} der 3D Dokumentation gegenüber der 2D Dokumentation
			\item
			\emph{Aufwand} zum Erstellen eines 3D Modells
			\item
			\emph{Qualität} und Fehler eines gewonnen Modells
			\item
			\emph{Integration} anderer Dokumentationen, insbesondere GIS-Daten
		\end{itemize}
	
	\section{Kriterien} \label{frag:kriterien}
		Einer Diskussion aufgrund praktischer Resultat muss eine theoretische Reflexion der Möglichkeiten und Präzisierung der Fragestellung und  der Kriterien zur Auswertung vorangehen.
		Die Motivation für die Nutzung von SfM in der Archäologie ist eine Verbesserung der Dokumentation. Es muss allem voran erst
		\subsection{Mehrwert}
			Der Mehrwert einer 3D Rekonstruktion hängt in erster Linie von dessen Qualität ab, je exakter das Modell, desto mehr Informationen sind darin dokumentiert.
			
			Wenig detaillierte Modell visualisieren geometrischen Zusammenhänge, Dimension, Orientierung und grob die farblichen Unterschiebe innerhalb einer Grabung. Siehe [TODO:Bild]. Die räumliche Darstellung vereinfacht es sich einen Überblick zu verschaffen und weitere Fragmente der Dokumentation zueinander in Bezug zu setzten.
			Zudem lassen sich 3D Modell gut für Öffentlichkeitsarbeit verwenden, da sie auch dem ungeübten Betrachter einfach und intuitiv die Situation aufzeigen.
			
			Bei höherer Qualität der Rekonstruktion sind auch Details wie Mauerfugen und kleinere Artefakte zu sehen. Siehe [TODO:Bild]. Als Dokumentation, im Sinne vom Erfassen möglichst vieler und genauer Informationen, ist dies eine sehr kompaktes und praktisches Format, da es neben den visuellen Informationen einer Fotografie auch die metrischen Informationen eines Planes und die geometrische Struktur mehrerer Skizzen vereint. Die Schwierigkeit ist dabei, dass der Detailgrad nicht frei bestimmt werden kann, sondern von der Qualität der Eingabebilder abhängt.
			
		\subsection{Aufwand}
		\subsection{Qualität}
		\subsection{Integration}
		
	\section{DRONARCH}
		Als technische Grundlage für die Fallstudien dient die Software \emph{DRONARCH} \cite{dronarch:github}, die im Rahmen dieser Arbeit geschrieben wurde und das SfM Verfahren verwendet.
		Im Gegensatz zu bestehenden Publikationen, die sich mit SfM in der Archäologie beschäftigen \cite{arch:laser_vs_dense_stereo, ARCM:ARCM667, ARP:ARP399, TUW-210216, DeReu20131108, the_cave, altai}, verwendet und publiziert \emph{DRONARCH} nur Open Source Software und will ein Werkzeug bieten mit wenig Geld und Zeitaufwand 3D Modell für den archäologischen Einsatz zu erstellen.

	