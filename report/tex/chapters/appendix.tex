\chapter{Begriffe}
		\section{Point Cloud}\label{app:point_cloud}
		Eine Point Cloud (Punktwolke) ist eine Menge von Punkten im 3D Raum. Die Punkte sind nicht miteinander verbunden, noch enthalten sie Informationen über Orientierung oder benachbarte Punkte. Die meisten 3D Scanner produzieren Point Clouds, die zu einem Mesh weiterverarbeitet werden können.
		Sind die Punkte dicht beieinander, spricht man von einer \emph{dense} (dicht) Point Cloud. Ansonsten nennt man sie \emph{sparse} (licht, locker).
		
		\section{Mesh}\label{app:mesh}
		Verbindet man mehrere Punkte zu einer Fläche, meist zu Dreiecken, enthält man ein Mesh. Dies hat eine klare Orientierung und setzt Punkte in Verbindung mit ihren Nachbarn. Enthält ein Mesh keine Löcher, nennt man es watertight (wasserdicht).
		
		\section{Orthofoto} \label{app:orthofoto}
		
		\section{Digitales Höhenmodell} \label{app:dtm}

\chapter{DRONARCH verwenden}
	\section{Tipps zum Aufnehmen von Fotos}\label{app:tip_foto}