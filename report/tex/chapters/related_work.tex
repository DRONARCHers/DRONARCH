\chapter{Aktueller Forschungsstand} \label{related:work}

	Unter den Sammelbegriffen \emph{Computer Applications and Quantitative Methods in Archaeology} (CAA) und \emph{Computational} oder \emph{Digital Archaeology} haben verschiedene Verfahren aus der Informatik, insbesondere der Geoinformationssysteme (GIS), Computer Vision und Photography Anwendung in der Archäologie gefunden.
	Man experimentiert mit automatischer Auswertung von Luft- und Satellitenbildern und verwendete GIS zur Lokalisierung und Registrierung von Fundorten\citeu{comp_app_arch}, arbeitet mit unbemannten Luftfahrzeugen und Wärmebildkameras\citeu{Casana2014207} und verwendet Helikites für hoch aufgelöste Luftaufnahmen\citeu{ARCM:ARCM667}.
	Die primäre Motivation dieser Forschung ist insgesamt eine exaktere und schnellere Dokumentation und eine umfänglichere Prospektion zu erreichen.

	\section{3D Scans}
		Durch die Entwicklung von Verfahren aus der Computer Vision Forschung ist es möglich auf zahlreiche Arten berührungs- und zerstörungsfreie 3D Scans von Objekten zu erstellen.
		\paragraph{Laser Scanner}
			Mit Laser Scannern kann ein sehr präzises Modell eines Objektes, oder auch einer ganzen Grabung gemacht werden. Galeazzi  \etal\citeu{arch:laser_vs_dense_stereo} testen dieses Verfahren in Höhlen und im Wald und beschreiben einen grossen Aufwand beim Installieren und Betrieben des verwendeten Scanners. Die Rechenzeit am Computer beträgt mehrere Stunden pro Scan und generiert als Resultat ein sehr detailliertes Mesh\autorefu{app:mesh}.
			Briese \etal\citeu{TUW-210216} verwenden einen mobilen auf einem Auto montierten Scanner um das Heidentor in Carnuntum (Österreich) zu erfassen, ein Verfahren, das in den meisten Fällen schlecht eingesetzt werden kann.
			In beiden Fällen ist die Ausrüstung und die dazugehörige Software teuer und erfordert ein gewisses Mass an Fachwissen zur korrekten Bedienung.

		\paragraph{Structure from Motion und Multiview Stereo}
			Durch die Verbindung der beiden Verfahren \emph{Structure from Motion}\autorefu{sfm} (SfM) und \emph{Multiview Stereo}\autorefu{mvs} (MvS) können aus unsortierten, nicht beschrifteten und nicht kalibrierten Bildern ohne grosse Nutzereingabe 3D Modelle erstellt werden \citeu{Szeliski:2010:CVA:1941882, Agarwal:2011:BRD:2001269.2001293, Furu:2010:PMVS}. Diese Verfahren wurden schon verwendet um mit Bildern aus dem Internet Monumente und ganze Städte zu rekonstruieren \citeu{Agarwal:2011:BRD:2001269.2001293, Furu:2010,  Snavely:2006:PTE:1179352.1141964}.
		
			De Reu \etal\citeu{DeReu20131108} verwendeten dieses Verfahren für die Dokumentation verschiedener Grabungen und konnten somit neben 3D Modellen auch Orthofoto\autorefu{app:orthofoto} und digitale Höhenmodelle\autorefu{app:dtm} (DTM) erstellen. Sie beziffern den Fehler ihrer Modelle mit einem RMSE (root-mean-square error) von meist weniger als 10cm.
			
			Plets \etal\citeu{altai} verwenden ebenfalls von Hand erstellte Fotos für die 3D Dokumentation von über 300 Petroglyphen im Altai Gebirge. Sie beschreiben das Verfahren als leicht anzuwenden und exakt und verwenden ebenfalls DTMs.
			
			Aus der einzigen veröffentlichten Videoaufnahme einer Kammer der grossen Pyramide des Königs Khufu, rekonstruieren Kawae \etal\citeu{the_cave} eine \emph{Dense Point Cloud}\autorefu{app:point_cloud}. Das verwendete Filmmaterial ist aus einer TV Produktion und zeigt meist den Moderator und das Innere der Kammer im Hintergrund. Trotz des ungeeigneten Bildmaterials gelingt es ihnen aufgrund des 3D Modells genaue Pläne der Kammer zu erstellen.
			
			Verhoeven \etal\citeu{ARP:ARP399, ARCM:ARCM667} arbeiten mit Luftaufnahmen als Grundlage und erstellen damit 3D Modelle, die sie mittels einzelner Vermessungspunkte orientieren und Skalieren. Durch diese Positionierung können die Modell mit bestehenden GIS in Zusammenhang gebracht werden.
			
			Brise \etal\citeu{TUW-210216} und Galeazzi \etal\citeu{arch:laser_vs_dense_stereo} vergleichen SfM und MvS mit Laser Scannern und kommen zum Schluss, dass Ersteres durch seine Flexibilität, sein nahtloses Einfügen in bestehende Grabungspraxis und seine zeit- und kostengünstige Anwendung eine ernst zu nehmende Alternative zu Laser Scannern sind. Sie beschreiben aber auch Szenarien in denen SfM und MvS schlecht oder gar nicht funktionieren und beobachten eine ungleichmässige Qualität in den erreichten Modellen.